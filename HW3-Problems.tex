\documentclass{article}
\usepackage{listings,amsmath}

%Listings Settings
\lstset{frame=tb,
language=bash,
aboveskip=3mm,
belowskip=3mm,
showstringspaces=false,
columns=flexible,
basicstyle={\ttfamily}
}

%Custom Commands
\newcommand{\git}{{\texttt{git}}}
\newcommand{\github}{{\texttt{Github}}}
\newcommand{\Python}{{\texttt{Python}}}
\newcommand{\python}{{\texttt{python}}}
\newcommand{\Anaconda}{{\texttt{Anaconda}}}
\newcommand*\diff{\mathop{}\!\mathrm{d}}
\newcommand*\Diff[1]{\mathop{}\!\mathrm{d^#1}}

\begin{document}

\begin{center}

\vspace*{-2.5cm}
\LARGE
\bf{Homework 3}
\vspace{1cm}

\large{Due: Sept 28, 2018}
\vspace{1cm}

\end{center}

Write python functions to implement the FTCS (forward in time, centered in space) method for solving a 1D scalar advection equation
\begin{equation*}
   \frac{\partial u}{\partial t} = -v \frac{\partial u}{\partial x}
\end{equation*}
where $u$ is a scalar and $v$ is constant. We showed in lecture that the basic approach
\begin{equation*}
   u_j^{n+1} = u^n_j - \frac{v \Delta t}{2\Delta x} \left( u^n_{j+1} - u^n_{j-1} \right) 
\end{equation*}
is unconditionally \emph{unstable}. Test this by running the FTCS scheme on a Gaussian Wave of varying $\sigma$ (e.g. $\sigma \gg \mathrm{d}x$ and $\sigma \sim \mathrm{d}x$ ) and show the instabilities growing over time.  Then, repeat your analysis using the corrected and \emph{stable} Lax-Friedrich Method and compare your results with those from FTCS.  What is the restriction on $\Delta t$ such that the Lax-Friedrich method remains stable?  Show what happens if you violate this condition.
\par
Write a \LaTeX{} report discussing your results.  It should include a short explanation of the algorithms and implementation with all relevant formula and a discusion of the following:
\begin{itemize}
   \item Plots showing growing instabilities in the Gaussian Wave for the FTCS scheme for a few different $\sigma$.
   \item The same plots from the Lax-Friedrich method including what happens when the Courant Condition is violated.
\end{itemize}
Include the \texttt{.pdf} version of the report and all Python files in your homework repo.
\par
\vspace{1cm}
\textbf{Bonus}: If you're feeling enthusiastic, try both of your advection schemes on a Gaussian Wavepacket!




\end{document}
